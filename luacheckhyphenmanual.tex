\documentclass{ltxdockit}
\usepackage{csquotes,graphicx}
\newcommand\pkgversion{see Makefile}

\titlepage{%
  title={The lua-check-hyphen package},
  subtitle={Checking your hyphenated words},
  url={https://github.com/pgundlach/lua-check-hyphen},
  author={Patrick Gundlach},
  email={patrick@gundla.ch},
  revision={\pkgversion},
  date={\today}}

\hypersetup{%
  pdftitle={The lua-check-hyphen package},
  pdfsubject={Checking your hyphenated words},
  pdfauthor={Patrick Gundlach},
  pdfkeywords={tex, latex, hyphenation}}




\begin{document}
\printtitlepage
\tableofcontents

\section{Introduction}

\TeX\ is pretty good at hyphenating words. But a computer algorithm is not perfect, sometimes we wish to manually check all used hyphenations in a document. This package helps you on this task.

\section{Usage}

This package only works with Lua\TeX. To activate it a

\begin{verbatim}
	\usepackage{lua-check-hyphen}
\end{verbatim}
is enough.

When you process your document with Lua\LaTeX, a file with the suffix \verb|.uhy| is created with that contains every hyphenated word. You can check that file for hyphenation errors. With the package option \texttt{nofile}, you can tell the package not to write the external file. In any case the list is written to the log file.

\section{Advanced usage}
\label{sec:advanced_usage}

With longer documents it becomes tedious checking for the same hyphenated words over and over. Therefore this package allows you to create a whitelist of \emph{accepted hyphenation}. This whitelist is a simple textfile where the words are separated by whitespace and the allowed hyphenation is marked with a dash (-). You can use the output file \texttt{.uhy} as an input for the whitelist file. You load the file in your document with

\begin{verbatim}
	\LuaCheckHyphen{whitelist=myfile.txt}
\end{verbatim}

and a sample file looks like this:

\begin{verbatim}
seren-ity
morn-ings
posses-sion
\end{verbatim}

or

\begin{verbatim}
seren-ity morn-ings posses-sion
\end{verbatim}

The format is not important, the entries must be separated by whitespace.

Words in this whitelist won't be reported. \sty{lua-check-hyphen} assumes that these are valid break points.

You can have more than one whitelist file by giving the filenames separated by comma:

\begin{verbatim}
\LuaCheckHyphen{whitelist={file1.txt,file2.txt,file3.txt}}
\end{verbatim}

\section{Visual marker}

You might want to use the PDF viewer to check for incorrectly hyphenated words. To enable visual feedback, just pass the \texttt{mark} option to the \cmd{usepackage} or to the \cmd{LuaCheckHyphen} command:

\begin{verbatim}
\usepackage[mark]{lua-check-hyphen}
% or:
\LuaCheckHyphen{mark}
\end{verbatim}

This marks all non-whitelisted used hyphenation points in the document with a black square (not to be confused with overfull boxes):

\includegraphics{sample-crop}

\section{Options to \cmd{LuaCheckHyphen}}

For the true/false options you don't need to write the \texttt{=true} part. All options except for the \texttt{whitelist} can be passed directly to the package:

\begin{verbatim}
\usepackage[mark,nofile]{lua-check-hyphen}
\end{verbatim}

\noindent All options can be set in \cmd{LuaCheckHyphen}.

\begin{verbatim}
\LuaCheckHyphen{mark,nofile,whitelist=myfile.txt}
\end{verbatim}


\begin{optionlist}
  \boolitem[true]{compact}{Words with different hyphenation occurrences are listed only once.}
  \boolitem[false]{mark}{Create visual markers when an unknown hyphenation encountered.}
  \boolitem[false]{nofile}{Don't write to an external file (\texttt{.uhy}).}
  \boolitem[false]{file}{Don't do anything. This is for the last document run, no marks will be generated.}
  \valitem{whitelist}{files}{List of files, separated by \enquote{\texttt{,}}. You need to wrap the filenames in braces \texttt{\textbraceleft...\textbraceright} so that the comma won't be seen as a option separator.}
\end{optionlist}

\section{Final document}

When you pass the Option \opt{final} to the \cmd{documentclass}, the \cmd{usepackage} or the \cmd{LuaCheckHyphen} command, all processing is disabled.

\section{Changes}
\begin{changelog}
\begin{release}{0.7}{2018-04-16}
  \item Take \texttt{--output-directory} into account.
\end{release}
\begin{release}{0.6}{2018-02-07}
  \item Remove dependencies and update package to a modern LuaTeX. Thanks to Christian Stadelmann.
\end{release}
\begin{release}{0.5}{2017-07-31}
  \item Fix error where I can't find a whitelist file, but the io.open() call does not return an error message.
\end{release}
\begin{release}{0.4}{2016-04-01}
  \item Compact representation of hyphenated words that occur more than once with different hyphenation positions (option copmact=true).
  \item Sort uhy entries alphabetically and only use lowercase words in the file.
\end{release}
 \begin{release}{0.2}{2012-06-22}
  \item Fix ligature counting error (github issue \#1)
  \item Add nofile option
  \item No hyphenation list in the terminal output (log only)
  \item License details added
 \end{release}
\end{changelog}

\section{Bugs}

\begin{itemize}
  \item Paragraphs that end with \verb+\end{document}+ are ignored. Inserting an empty line or
        \cmd{par} helps.
\end{itemize}

\section{License}

This file (documentation and the package) is released under the terms of the MIT License. See the file \texttt{mit-license.txt} in the distribution.

% Copyright (c) 2012 Patrick Gundlach

% Permission is hereby granted, free of charge, to any person obtaining a copy
% of this software and associated documentation files (the \enquote{Software}), to deal
% in the Software without restriction, including without limitation the rights
% to use, copy, modify, merge, publish, distribute, sublicense, and/or sell
% copies of the Software, and to permit persons to whom the Software is
% furnished to do so, subject to the following conditions:

% The above copyright notice and this permission notice shall be included in all
% copies or substantial portions of the Software.

% THE SOFTWARE IS PROVIDED \enquote{AS IS}, WITHOUT WARRANTY OF ANY KIND, EXPRESS OR
% IMPLIED, INCLUDING BUT NOT LIMITED TO THE WARRANTIES OF MERCHANTABILITY,
% FITNESS FOR A PARTICULAR PURPOSE AND NONINFRINGEMENT. IN NO EVENT SHALL THE
% AUTHORS OR COPYRIGHT HOLDERS BE LIABLE FOR ANY CLAIM, DAMAGES OR OTHER
% LIABILITY, WHETHER IN AN ACTION OF CONTRACT, TORT OR OTHERWISE, ARISING FROM,
% OUT OF OR IN CONNECTION WITH THE SOFTWARE OR THE USE OR OTHER DEALINGS IN THE
% SOFTWARE.

\end{document}