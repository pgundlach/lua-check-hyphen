\documentclass{ltxdockit}
\usepackage{dtklogos}

\titlepage{%
  title={The lua-check-hyphen package},
  subtitle={Checking your hyphenated words},
  url={https://github.com/pgundlach/lua-check-hyphen},
  author={Patrick Gundlach},
  email={patrick@gundla.ch},
  revision={0.1},
  date={\today}}

\hypersetup{%
  pdftitle={The lua-check-hyphen package},
  pdfsubject={Checking your hyphenated words},
  pdfauthor={Patrick Gundlach},
  pdfkeywords={tex, latex, hyphenation}}




\begin{document}
\printtitlepage
\tableofcontents

\section{Introduction}

\TeX\ is pretty good at hyphenating words. But a computer algorithm is not perfect, sometimes we wish to manually check all used hyphenations in a document. This package helps you on this task.

\section{Usage}

This package only works with \LuaTeX. To activate it a

\begin{verbatim}
	\usepackage{lua-check-hyphen}
\end{verbatim}
is enough.

When you process your document with \LuaLaTeX, a file with the suffix \verb|.uhy| is created with that contains every hyphenated word. You can check that file for hyphenation errors.

\section{Advanced usage}
\label{sec:advanced_usage}

With longer documents it becomes tedious checking for the same hyphenated words over and over. Therefore this package allows you to create a whitelist of \emph{accepted hyphenation}. This whitelist is a simple textfile where the words are separated by whitespace and the allowed hyphenation is marked with a dash (-). You can use the output file \texttt{.uhy} as an input for the whitelist file. You load the file in your document with

\begin{verbatim}
	\LuaCheckHyphen{whitelist=myfile.txt}
\end{verbatim}

and a sample file looks like this:

\begin{verbatim}
seren-ity
morn-ings
posses-sion
\end{verbatim}

or

\begin{verbatim}
seren-ity morn-ings posses-sion
\end{verbatim}

The format is not important, the entries must be separated by whitespace.

Words in this whitelist won't be reported. \sty{lua-check-hyphen} assumes that these are valid break points.

You can have more than one whitelist file by giving the filenames separated by comma:

\begin{verbatim}
	\LuaCheckHyphen{whitelist={file1.txt,file2.txt,file3.txt}}
\end{verbatim}

\section{Visual marker}

You might want to use the PDF viewer to check for incorrectly hyphenated words. To enable visual feedback, just pass the \texttt{mark} option to the \cmd{usepackage} or to the \cmd{LuaCheckHyphen} command:

\begin{verbatim}
\usepackage{lua-check-hyphen}
% or:
\LuaCheckHyphen{mark}
\end{verbatim}

This marks all non-whitelisted used hyphenation points in the document with a black square (not to be confused with overfull boxes): 

\includegraphics{sample-crop}

\section{Final document}

When you pass the Option \opt{final} to the \cmd{documentclass}, the \cmd{usepackage} or the \cmd{LuaCheckHyphen} command, all processing is disabled.

\end{document}